
% Default to the notebook output style

    


% Inherit from the specified cell style.




    
\documentclass[11pt]{article}

    
    
    \usepackage[T1]{fontenc}
    % Nicer default font (+ math font) than Computer Modern for most use cases
    \usepackage{mathpazo}

    % Basic figure setup, for now with no caption control since it's done
    % automatically by Pandoc (which extracts ![](path) syntax from Markdown).
    \usepackage{graphicx}
    % We will generate all images so they have a width \maxwidth. This means
    % that they will get their normal width if they fit onto the page, but
    % are scaled down if they would overflow the margins.
    \makeatletter
    \def\maxwidth{\ifdim\Gin@nat@width>\linewidth\linewidth
    \else\Gin@nat@width\fi}
    \makeatother
    \let\Oldincludegraphics\includegraphics
    % Set max figure width to be 80% of text width, for now hardcoded.
    \renewcommand{\includegraphics}[1]{\Oldincludegraphics[width=.8\maxwidth]{#1}}
    % Ensure that by default, figures have no caption (until we provide a
    % proper Figure object with a Caption API and a way to capture that
    % in the conversion process - todo).
    \usepackage{caption}
    \DeclareCaptionLabelFormat{nolabel}{}
    \captionsetup{labelformat=nolabel}

    \usepackage{adjustbox} % Used to constrain images to a maximum size 
    \usepackage{xcolor} % Allow colors to be defined
    \usepackage{enumerate} % Needed for markdown enumerations to work
    \usepackage{geometry} % Used to adjust the document margins
    \usepackage{amsmath} % Equations
    \usepackage{amssymb} % Equations
    \usepackage{textcomp} % defines textquotesingle
    % Hack from http://tex.stackexchange.com/a/47451/13684:
    \AtBeginDocument{%
        \def\PYZsq{\textquotesingle}% Upright quotes in Pygmentized code
    }
    \usepackage{upquote} % Upright quotes for verbatim code
    \usepackage{eurosym} % defines \euro
    \usepackage[mathletters]{ucs} % Extended unicode (utf-8) support
    \usepackage[utf8x]{inputenc} % Allow utf-8 characters in the tex document
    \usepackage{fancyvrb} % verbatim replacement that allows latex
    \usepackage{grffile} % extends the file name processing of package graphics 
                         % to support a larger range 
    % The hyperref package gives us a pdf with properly built
    % internal navigation ('pdf bookmarks' for the table of contents,
    % internal cross-reference links, web links for URLs, etc.)
    \usepackage{hyperref}
    \usepackage{longtable} % longtable support required by pandoc >1.10
    \usepackage{booktabs}  % table support for pandoc > 1.12.2
    \usepackage[inline]{enumitem} % IRkernel/repr support (it uses the enumerate* environment)
    \usepackage[normalem]{ulem} % ulem is needed to support strikethroughs (\sout)
                                % normalem makes italics be italics, not underlines
    

    
    
    % Colors for the hyperref package
    \definecolor{urlcolor}{rgb}{0,.145,.698}
    \definecolor{linkcolor}{rgb}{.71,0.21,0.01}
    \definecolor{citecolor}{rgb}{.12,.54,.11}

    % ANSI colors
    \definecolor{ansi-black}{HTML}{3E424D}
    \definecolor{ansi-black-intense}{HTML}{282C36}
    \definecolor{ansi-red}{HTML}{E75C58}
    \definecolor{ansi-red-intense}{HTML}{B22B31}
    \definecolor{ansi-green}{HTML}{00A250}
    \definecolor{ansi-green-intense}{HTML}{007427}
    \definecolor{ansi-yellow}{HTML}{DDB62B}
    \definecolor{ansi-yellow-intense}{HTML}{B27D12}
    \definecolor{ansi-blue}{HTML}{208FFB}
    \definecolor{ansi-blue-intense}{HTML}{0065CA}
    \definecolor{ansi-magenta}{HTML}{D160C4}
    \definecolor{ansi-magenta-intense}{HTML}{A03196}
    \definecolor{ansi-cyan}{HTML}{60C6C8}
    \definecolor{ansi-cyan-intense}{HTML}{258F8F}
    \definecolor{ansi-white}{HTML}{C5C1B4}
    \definecolor{ansi-white-intense}{HTML}{A1A6B2}

    % commands and environments needed by pandoc snippets
    % extracted from the output of `pandoc -s`
    \providecommand{\tightlist}{%
      \setlength{\itemsep}{0pt}\setlength{\parskip}{0pt}}
    \DefineVerbatimEnvironment{Highlighting}{Verbatim}{commandchars=\\\{\}}
    % Add ',fontsize=\small' for more characters per line
    \newenvironment{Shaded}{}{}
    \newcommand{\KeywordTok}[1]{\textcolor[rgb]{0.00,0.44,0.13}{\textbf{{#1}}}}
    \newcommand{\DataTypeTok}[1]{\textcolor[rgb]{0.56,0.13,0.00}{{#1}}}
    \newcommand{\DecValTok}[1]{\textcolor[rgb]{0.25,0.63,0.44}{{#1}}}
    \newcommand{\BaseNTok}[1]{\textcolor[rgb]{0.25,0.63,0.44}{{#1}}}
    \newcommand{\FloatTok}[1]{\textcolor[rgb]{0.25,0.63,0.44}{{#1}}}
    \newcommand{\CharTok}[1]{\textcolor[rgb]{0.25,0.44,0.63}{{#1}}}
    \newcommand{\StringTok}[1]{\textcolor[rgb]{0.25,0.44,0.63}{{#1}}}
    \newcommand{\CommentTok}[1]{\textcolor[rgb]{0.38,0.63,0.69}{\textit{{#1}}}}
    \newcommand{\OtherTok}[1]{\textcolor[rgb]{0.00,0.44,0.13}{{#1}}}
    \newcommand{\AlertTok}[1]{\textcolor[rgb]{1.00,0.00,0.00}{\textbf{{#1}}}}
    \newcommand{\FunctionTok}[1]{\textcolor[rgb]{0.02,0.16,0.49}{{#1}}}
    \newcommand{\RegionMarkerTok}[1]{{#1}}
    \newcommand{\ErrorTok}[1]{\textcolor[rgb]{1.00,0.00,0.00}{\textbf{{#1}}}}
    \newcommand{\NormalTok}[1]{{#1}}
    
    % Additional commands for more recent versions of Pandoc
    \newcommand{\ConstantTok}[1]{\textcolor[rgb]{0.53,0.00,0.00}{{#1}}}
    \newcommand{\SpecialCharTok}[1]{\textcolor[rgb]{0.25,0.44,0.63}{{#1}}}
    \newcommand{\VerbatimStringTok}[1]{\textcolor[rgb]{0.25,0.44,0.63}{{#1}}}
    \newcommand{\SpecialStringTok}[1]{\textcolor[rgb]{0.73,0.40,0.53}{{#1}}}
    \newcommand{\ImportTok}[1]{{#1}}
    \newcommand{\DocumentationTok}[1]{\textcolor[rgb]{0.73,0.13,0.13}{\textit{{#1}}}}
    \newcommand{\AnnotationTok}[1]{\textcolor[rgb]{0.38,0.63,0.69}{\textbf{\textit{{#1}}}}}
    \newcommand{\CommentVarTok}[1]{\textcolor[rgb]{0.38,0.63,0.69}{\textbf{\textit{{#1}}}}}
    \newcommand{\VariableTok}[1]{\textcolor[rgb]{0.10,0.09,0.49}{{#1}}}
    \newcommand{\ControlFlowTok}[1]{\textcolor[rgb]{0.00,0.44,0.13}{\textbf{{#1}}}}
    \newcommand{\OperatorTok}[1]{\textcolor[rgb]{0.40,0.40,0.40}{{#1}}}
    \newcommand{\BuiltInTok}[1]{{#1}}
    \newcommand{\ExtensionTok}[1]{{#1}}
    \newcommand{\PreprocessorTok}[1]{\textcolor[rgb]{0.74,0.48,0.00}{{#1}}}
    \newcommand{\AttributeTok}[1]{\textcolor[rgb]{0.49,0.56,0.16}{{#1}}}
    \newcommand{\InformationTok}[1]{\textcolor[rgb]{0.38,0.63,0.69}{\textbf{\textit{{#1}}}}}
    \newcommand{\WarningTok}[1]{\textcolor[rgb]{0.38,0.63,0.69}{\textbf{\textit{{#1}}}}}
    
    
    % Define a nice break command that doesn't care if a line doesn't already
    % exist.
    \def\br{\hspace*{\fill} \\* }
    % Math Jax compatability definitions
    \def\gt{>}
    \def\lt{<}
    % Document parameters
    \title{Text\_Clustering}
    
    
    

    % Pygments definitions
    
\makeatletter
\def\PY@reset{\let\PY@it=\relax \let\PY@bf=\relax%
    \let\PY@ul=\relax \let\PY@tc=\relax%
    \let\PY@bc=\relax \let\PY@ff=\relax}
\def\PY@tok#1{\csname PY@tok@#1\endcsname}
\def\PY@toks#1+{\ifx\relax#1\empty\else%
    \PY@tok{#1}\expandafter\PY@toks\fi}
\def\PY@do#1{\PY@bc{\PY@tc{\PY@ul{%
    \PY@it{\PY@bf{\PY@ff{#1}}}}}}}
\def\PY#1#2{\PY@reset\PY@toks#1+\relax+\PY@do{#2}}

\expandafter\def\csname PY@tok@w\endcsname{\def\PY@tc##1{\textcolor[rgb]{0.73,0.73,0.73}{##1}}}
\expandafter\def\csname PY@tok@c\endcsname{\let\PY@it=\textit\def\PY@tc##1{\textcolor[rgb]{0.25,0.50,0.50}{##1}}}
\expandafter\def\csname PY@tok@cp\endcsname{\def\PY@tc##1{\textcolor[rgb]{0.74,0.48,0.00}{##1}}}
\expandafter\def\csname PY@tok@k\endcsname{\let\PY@bf=\textbf\def\PY@tc##1{\textcolor[rgb]{0.00,0.50,0.00}{##1}}}
\expandafter\def\csname PY@tok@kp\endcsname{\def\PY@tc##1{\textcolor[rgb]{0.00,0.50,0.00}{##1}}}
\expandafter\def\csname PY@tok@kt\endcsname{\def\PY@tc##1{\textcolor[rgb]{0.69,0.00,0.25}{##1}}}
\expandafter\def\csname PY@tok@o\endcsname{\def\PY@tc##1{\textcolor[rgb]{0.40,0.40,0.40}{##1}}}
\expandafter\def\csname PY@tok@ow\endcsname{\let\PY@bf=\textbf\def\PY@tc##1{\textcolor[rgb]{0.67,0.13,1.00}{##1}}}
\expandafter\def\csname PY@tok@nb\endcsname{\def\PY@tc##1{\textcolor[rgb]{0.00,0.50,0.00}{##1}}}
\expandafter\def\csname PY@tok@nf\endcsname{\def\PY@tc##1{\textcolor[rgb]{0.00,0.00,1.00}{##1}}}
\expandafter\def\csname PY@tok@nc\endcsname{\let\PY@bf=\textbf\def\PY@tc##1{\textcolor[rgb]{0.00,0.00,1.00}{##1}}}
\expandafter\def\csname PY@tok@nn\endcsname{\let\PY@bf=\textbf\def\PY@tc##1{\textcolor[rgb]{0.00,0.00,1.00}{##1}}}
\expandafter\def\csname PY@tok@ne\endcsname{\let\PY@bf=\textbf\def\PY@tc##1{\textcolor[rgb]{0.82,0.25,0.23}{##1}}}
\expandafter\def\csname PY@tok@nv\endcsname{\def\PY@tc##1{\textcolor[rgb]{0.10,0.09,0.49}{##1}}}
\expandafter\def\csname PY@tok@no\endcsname{\def\PY@tc##1{\textcolor[rgb]{0.53,0.00,0.00}{##1}}}
\expandafter\def\csname PY@tok@nl\endcsname{\def\PY@tc##1{\textcolor[rgb]{0.63,0.63,0.00}{##1}}}
\expandafter\def\csname PY@tok@ni\endcsname{\let\PY@bf=\textbf\def\PY@tc##1{\textcolor[rgb]{0.60,0.60,0.60}{##1}}}
\expandafter\def\csname PY@tok@na\endcsname{\def\PY@tc##1{\textcolor[rgb]{0.49,0.56,0.16}{##1}}}
\expandafter\def\csname PY@tok@nt\endcsname{\let\PY@bf=\textbf\def\PY@tc##1{\textcolor[rgb]{0.00,0.50,0.00}{##1}}}
\expandafter\def\csname PY@tok@nd\endcsname{\def\PY@tc##1{\textcolor[rgb]{0.67,0.13,1.00}{##1}}}
\expandafter\def\csname PY@tok@s\endcsname{\def\PY@tc##1{\textcolor[rgb]{0.73,0.13,0.13}{##1}}}
\expandafter\def\csname PY@tok@sd\endcsname{\let\PY@it=\textit\def\PY@tc##1{\textcolor[rgb]{0.73,0.13,0.13}{##1}}}
\expandafter\def\csname PY@tok@si\endcsname{\let\PY@bf=\textbf\def\PY@tc##1{\textcolor[rgb]{0.73,0.40,0.53}{##1}}}
\expandafter\def\csname PY@tok@se\endcsname{\let\PY@bf=\textbf\def\PY@tc##1{\textcolor[rgb]{0.73,0.40,0.13}{##1}}}
\expandafter\def\csname PY@tok@sr\endcsname{\def\PY@tc##1{\textcolor[rgb]{0.73,0.40,0.53}{##1}}}
\expandafter\def\csname PY@tok@ss\endcsname{\def\PY@tc##1{\textcolor[rgb]{0.10,0.09,0.49}{##1}}}
\expandafter\def\csname PY@tok@sx\endcsname{\def\PY@tc##1{\textcolor[rgb]{0.00,0.50,0.00}{##1}}}
\expandafter\def\csname PY@tok@m\endcsname{\def\PY@tc##1{\textcolor[rgb]{0.40,0.40,0.40}{##1}}}
\expandafter\def\csname PY@tok@gh\endcsname{\let\PY@bf=\textbf\def\PY@tc##1{\textcolor[rgb]{0.00,0.00,0.50}{##1}}}
\expandafter\def\csname PY@tok@gu\endcsname{\let\PY@bf=\textbf\def\PY@tc##1{\textcolor[rgb]{0.50,0.00,0.50}{##1}}}
\expandafter\def\csname PY@tok@gd\endcsname{\def\PY@tc##1{\textcolor[rgb]{0.63,0.00,0.00}{##1}}}
\expandafter\def\csname PY@tok@gi\endcsname{\def\PY@tc##1{\textcolor[rgb]{0.00,0.63,0.00}{##1}}}
\expandafter\def\csname PY@tok@gr\endcsname{\def\PY@tc##1{\textcolor[rgb]{1.00,0.00,0.00}{##1}}}
\expandafter\def\csname PY@tok@ge\endcsname{\let\PY@it=\textit}
\expandafter\def\csname PY@tok@gs\endcsname{\let\PY@bf=\textbf}
\expandafter\def\csname PY@tok@gp\endcsname{\let\PY@bf=\textbf\def\PY@tc##1{\textcolor[rgb]{0.00,0.00,0.50}{##1}}}
\expandafter\def\csname PY@tok@go\endcsname{\def\PY@tc##1{\textcolor[rgb]{0.53,0.53,0.53}{##1}}}
\expandafter\def\csname PY@tok@gt\endcsname{\def\PY@tc##1{\textcolor[rgb]{0.00,0.27,0.87}{##1}}}
\expandafter\def\csname PY@tok@err\endcsname{\def\PY@bc##1{\setlength{\fboxsep}{0pt}\fcolorbox[rgb]{1.00,0.00,0.00}{1,1,1}{\strut ##1}}}
\expandafter\def\csname PY@tok@kc\endcsname{\let\PY@bf=\textbf\def\PY@tc##1{\textcolor[rgb]{0.00,0.50,0.00}{##1}}}
\expandafter\def\csname PY@tok@kd\endcsname{\let\PY@bf=\textbf\def\PY@tc##1{\textcolor[rgb]{0.00,0.50,0.00}{##1}}}
\expandafter\def\csname PY@tok@kn\endcsname{\let\PY@bf=\textbf\def\PY@tc##1{\textcolor[rgb]{0.00,0.50,0.00}{##1}}}
\expandafter\def\csname PY@tok@kr\endcsname{\let\PY@bf=\textbf\def\PY@tc##1{\textcolor[rgb]{0.00,0.50,0.00}{##1}}}
\expandafter\def\csname PY@tok@bp\endcsname{\def\PY@tc##1{\textcolor[rgb]{0.00,0.50,0.00}{##1}}}
\expandafter\def\csname PY@tok@fm\endcsname{\def\PY@tc##1{\textcolor[rgb]{0.00,0.00,1.00}{##1}}}
\expandafter\def\csname PY@tok@vc\endcsname{\def\PY@tc##1{\textcolor[rgb]{0.10,0.09,0.49}{##1}}}
\expandafter\def\csname PY@tok@vg\endcsname{\def\PY@tc##1{\textcolor[rgb]{0.10,0.09,0.49}{##1}}}
\expandafter\def\csname PY@tok@vi\endcsname{\def\PY@tc##1{\textcolor[rgb]{0.10,0.09,0.49}{##1}}}
\expandafter\def\csname PY@tok@vm\endcsname{\def\PY@tc##1{\textcolor[rgb]{0.10,0.09,0.49}{##1}}}
\expandafter\def\csname PY@tok@sa\endcsname{\def\PY@tc##1{\textcolor[rgb]{0.73,0.13,0.13}{##1}}}
\expandafter\def\csname PY@tok@sb\endcsname{\def\PY@tc##1{\textcolor[rgb]{0.73,0.13,0.13}{##1}}}
\expandafter\def\csname PY@tok@sc\endcsname{\def\PY@tc##1{\textcolor[rgb]{0.73,0.13,0.13}{##1}}}
\expandafter\def\csname PY@tok@dl\endcsname{\def\PY@tc##1{\textcolor[rgb]{0.73,0.13,0.13}{##1}}}
\expandafter\def\csname PY@tok@s2\endcsname{\def\PY@tc##1{\textcolor[rgb]{0.73,0.13,0.13}{##1}}}
\expandafter\def\csname PY@tok@sh\endcsname{\def\PY@tc##1{\textcolor[rgb]{0.73,0.13,0.13}{##1}}}
\expandafter\def\csname PY@tok@s1\endcsname{\def\PY@tc##1{\textcolor[rgb]{0.73,0.13,0.13}{##1}}}
\expandafter\def\csname PY@tok@mb\endcsname{\def\PY@tc##1{\textcolor[rgb]{0.40,0.40,0.40}{##1}}}
\expandafter\def\csname PY@tok@mf\endcsname{\def\PY@tc##1{\textcolor[rgb]{0.40,0.40,0.40}{##1}}}
\expandafter\def\csname PY@tok@mh\endcsname{\def\PY@tc##1{\textcolor[rgb]{0.40,0.40,0.40}{##1}}}
\expandafter\def\csname PY@tok@mi\endcsname{\def\PY@tc##1{\textcolor[rgb]{0.40,0.40,0.40}{##1}}}
\expandafter\def\csname PY@tok@il\endcsname{\def\PY@tc##1{\textcolor[rgb]{0.40,0.40,0.40}{##1}}}
\expandafter\def\csname PY@tok@mo\endcsname{\def\PY@tc##1{\textcolor[rgb]{0.40,0.40,0.40}{##1}}}
\expandafter\def\csname PY@tok@ch\endcsname{\let\PY@it=\textit\def\PY@tc##1{\textcolor[rgb]{0.25,0.50,0.50}{##1}}}
\expandafter\def\csname PY@tok@cm\endcsname{\let\PY@it=\textit\def\PY@tc##1{\textcolor[rgb]{0.25,0.50,0.50}{##1}}}
\expandafter\def\csname PY@tok@cpf\endcsname{\let\PY@it=\textit\def\PY@tc##1{\textcolor[rgb]{0.25,0.50,0.50}{##1}}}
\expandafter\def\csname PY@tok@c1\endcsname{\let\PY@it=\textit\def\PY@tc##1{\textcolor[rgb]{0.25,0.50,0.50}{##1}}}
\expandafter\def\csname PY@tok@cs\endcsname{\let\PY@it=\textit\def\PY@tc##1{\textcolor[rgb]{0.25,0.50,0.50}{##1}}}

\def\PYZbs{\char`\\}
\def\PYZus{\char`\_}
\def\PYZob{\char`\{}
\def\PYZcb{\char`\}}
\def\PYZca{\char`\^}
\def\PYZam{\char`\&}
\def\PYZlt{\char`\<}
\def\PYZgt{\char`\>}
\def\PYZsh{\char`\#}
\def\PYZpc{\char`\%}
\def\PYZdl{\char`\$}
\def\PYZhy{\char`\-}
\def\PYZsq{\char`\'}
\def\PYZdq{\char`\"}
\def\PYZti{\char`\~}
% for compatibility with earlier versions
\def\PYZat{@}
\def\PYZlb{[}
\def\PYZrb{]}
\makeatother


    % Exact colors from NB
    \definecolor{incolor}{rgb}{0.0, 0.0, 0.5}
    \definecolor{outcolor}{rgb}{0.545, 0.0, 0.0}



    
    % Prevent overflowing lines due to hard-to-break entities
    \sloppy 
    % Setup hyperref package
    \hypersetup{
      breaklinks=true,  % so long urls are correctly broken across lines
      colorlinks=true,
      urlcolor=urlcolor,
      linkcolor=linkcolor,
      citecolor=citecolor,
      }
    % Slightly bigger margins than the latex defaults
    
    \geometry{verbose,tmargin=1in,bmargin=1in,lmargin=1in,rmargin=1in}
    
    

    \begin{document}
    
    
    \maketitle
    
    

    
    \#

Unsupervised Learing: Text Clustering

References: *
https://www-users.cs.umn.edu/\textsubscript{kumar/dmbook/ch8.pdf *
http://infolab.stanford.edu/}ullman/mmds/ch7.pdf

    \hypertarget{clustering-vs.classification}{%
\subsection{1. Clustering
vs.~Classification}\label{clustering-vs.classification}}

\begin{itemize}
\tightlist
\item
  Clustering (Unsupervised): divide a set of objects into clusters
  (parts of the set) so that objects in the same cluster are similar to
  each other, and/or objects in different clusters are dissimilar.

  \begin{itemize}
  \tightlist
  \item
    Representation of the objects
  \item
    Similarity/distance measure
  \end{itemize}
\item
  Classifification (Supervised): group objects into predetermined
  categories

  \begin{itemize}
  \tightlist
  \item
    Representation of the objects
  \item
    A training set
  \end{itemize}
\end{itemize}

    \hypertarget{why-clustering}{%
\subsection{2. Why clustering}\label{why-clustering}}

\begin{itemize}
\tightlist
\item
  Understand conceptually meaningful groups of objects that share common
  characteristics
\item
  Provides an abstraction from individual data objects to the clusters
  in which those data objects reside
\item
  Uses of clustering

  \begin{itemize}
  \tightlist
  \item
    Summarization
  \item
    Compression
  \item
    Efficiently finding nearest neighbors
  \end{itemize}
\end{itemize}

    \hypertarget{types-of-clusterings}{%
\subsection{3. Types of Clusterings}\label{types-of-clusterings}}

\begin{itemize}
\tightlist
\item
  Different kinds of models
  (https://www.geeksforgeeks.org/different-types-clustering-algorithm/):

  \begin{itemize}
  \tightlist
  \item
    Centroid models (partition):

    \begin{itemize}
    \tightlist
    \item
      Similarity is derived as the closeness of a data point to the
      centroid of clusters.
    \item
      Flat partition, e.g.~K-Means 
    \end{itemize}
  \item
    Connectivity models (Hierarchical algorithms):

    \begin{itemize}
    \tightlist
    \item
      Data points closer in data space exhibit more similarity to each
      other than the data points lying farther away.\\
    \item
      Hierarchy of clusters, e.g.~agglomerative clustering 
    \end{itemize}
  \item
    Distribution models:

    \begin{itemize}
    \tightlist
    \item
      How probable is it that all data points in the cluster belong to
      the same distribution, concept, or topic
    \item
      e.g.~Latent Semantics Analysis, Latent Dirichlet Allocation (LDA) 
    \end{itemize}
  \item
    Density models: clusters correspond to areas of varied density of
    data points in the data space

    \begin{itemize}
    \tightlist
    \item
      e.g.~DBSCAN 
    \end{itemize}
  \end{itemize}
\item
  Exclusive vs.~Overlapping

  \begin{itemize}
  \tightlist
  \item
    Exclusive: each object is assigned to a single cluster, e.g.~K-Means
  \item
    Overlapping (non-exclusive): an object can simultaneously belong to
    more than one cluster, e.g.~LDA
  \end{itemize}
\end{itemize}

    \hypertarget{evaluation-of-clustering-what-is-a-good-clustering}{%
\subsection{4. Evaluation of Clustering: What is a good
clustering}\label{evaluation-of-clustering-what-is-a-good-clustering}}

\hypertarget{external-evaluation}{%
\subsubsection{4.1 External Evaluation:}\label{external-evaluation}}

\begin{itemize}
\tightlist
\item
  External evaluation measures the degree to which predicted clustering
  labels correspond to actual class labels\\
\item
  \textbf{Precision} and \textbf{Recall}
\end{itemize}

    \hypertarget{internal-evaluation}{%
\subsubsection{4.2. Internal Evaluation}\label{internal-evaluation}}

 * \textbf{Cohension (Intra-cluster similarity)}: how ``cohesive'' a
cluster is, i.e.~the average similarity of objects in the same cluster.
- e.g.~cluster radius: \(\max{d(x, μ_A)}\) where \(μ_A\) is the
arithmetic mean of cluster A and \(x\) is a point in A - e.g.~cluster
diameter: \(\max{d(x, y)}\) where \(x,y\) are two points in cluster A

\begin{itemize}
\tightlist
\item
  \textbf{Separation (Inter-cluster dissimilarity)}: how ``separate'' a
  cluster from another, i.e.~the average similarity of all samples in
  cluster \(A\) to all the samples in cluster \(B\).

  \begin{itemize}
  \tightlist
  \item
    e.g.~Separation can be calculated as average distance:
    \(\frac{1}{|A|*|B|}\sum_{x \in A}{\sum_{y \in B}{d(x, y)}}\)
  \end{itemize}
\item
  Metrics with combined cohension and separation
  (http://scikit-learn.org/stable/modules/clustering.html\#clustering-evaluation)

  \begin{itemize}
  \tightlist
  \item
    Silhouette Coefficient: \(s=\frac{b-a}{\max(a,b)}\), where \(a\):
    the mean distance between a sample and all other points in the same
    cluster, and \(b\): the mean distance between a sample and all other
    points in the next nearest cluster. \(s \in [-1, 1]\).
  \item
    Calinski-Harabaz Index: \(s=\frac{b}{a}\) where \(a\) is mean
    within-cluster separation, and \(b\) is the mean between-cluster
    separation
  \end{itemize}
\end{itemize}

    \hypertarget{k-means}{%
\subsection{5. K-Means}\label{k-means}}

\hypertarget{algorithm-outline-cluster-objects-into-k-clusters}{%
\subsubsection{5.1. Algorithm outline: Cluster objects into K
clusters}\label{algorithm-outline-cluster-objects-into-k-clusters}}

 - Algorithm: 1. Select K points as initial centroids 2. Repeat until
centroids do not change: 1. Form K clusters by assigning each point to
its closest centroid by distance. 2. Recompute the centroid of each
cluster as the arithmetic mean of samples within the cluster. - A few
observations of K-means: - Initial centroids have an impact on
clustering. Usually, several rounds of clustering with random initial
centroids are performed, and the most commonly occurring output
centroids are chosen. - Centroids and distance measure are crtical in
the algorithm - \textbf{Euclidean distance}: - The best centroid for
minimizing the average distance from all samples to the centroid is the
mean of points in the cluster
(https://www-users.cs.umn.edu/\textasciitilde{}kumar001/dmbook/ch8.pdf)
- Curse of dimensionality - Sensitive to outliers - \textbf{Cosine
similarity}: * Well-accepted similarity measure for documents * It is
not guaranteed that the mean of samples in a cluster is the best
centroid * For text clustering, the centroid does not stand for an
actual document. How to interpret clusters? - A modified version of
Kmeans is called \textbf{K-medoids}, where a representative sample is
choosen as the center of a cluster, called as a medoid. - Python
packages for Kmean * NLTK: can choose Euclidean or Cosine similarity as
distance measure * Sklearn: only Euclidean distance is supported

    \begin{Verbatim}[commandchars=\\\{\}]
{\color{incolor}In [{\color{incolor}1}]:} \PY{k+kn}{from} \PY{n+nn}{IPython}\PY{n+nn}{.}\PY{n+nn}{core}\PY{n+nn}{.}\PY{n+nn}{interactiveshell} \PY{k}{import} \PY{n}{InteractiveShell}
        \PY{n}{InteractiveShell}\PY{o}{.}\PY{n}{ast\PYZus{}node\PYZus{}interactivity} \PY{o}{=} \PY{l+s+s2}{\PYZdq{}}\PY{l+s+s2}{all}\PY{l+s+s2}{\PYZdq{}}
\end{Verbatim}


    \begin{Verbatim}[commandchars=\\\{\}]
{\color{incolor}In [{\color{incolor}2}]:} \PY{c+c1}{\PYZsh{} Exercise 5.1.1 Load data and generate TF\PYZhy{}IDF}
        \PY{c+c1}{\PYZsh{} Load datasets (http://qwone.com/\PYZti{}jason/20Newsgroups/)}
        \PY{c+c1}{\PYZsh{} A subset is loaded}
        
        \PY{k+kn}{import} \PY{n+nn}{pandas} \PY{k}{as} \PY{n+nn}{pd}
        \PY{n}{df}\PY{o}{=}\PY{n}{pd}\PY{o}{.}\PY{n}{read\PYZus{}csv}\PY{p}{(}\PY{l+s+s2}{\PYZdq{}}\PY{l+s+s2}{../../dataset/twenty\PYZus{}news\PYZus{}data.csv}\PY{l+s+s2}{\PYZdq{}}\PY{p}{,}\PYZbs{}
                       \PY{n}{header}\PY{o}{=}\PY{l+m+mi}{0}\PY{p}{)}
        \PY{n}{df}\PY{o}{.}\PY{n}{head}\PY{p}{(}\PY{p}{)}
        
        \PY{c+c1}{\PYZsh{} get unique labels}
        \PY{n}{df}\PY{p}{[}\PY{l+s+s2}{\PYZdq{}}\PY{l+s+s2}{label}\PY{l+s+s2}{\PYZdq{}}\PY{p}{]}\PY{o}{.}\PY{n}{unique}\PY{p}{(}\PY{p}{)}
        
        \PY{c+c1}{\PYZsh{} Select three labels for now}
        \PY{n}{labels} \PY{o}{=}\PY{p}{[}\PY{l+s+s1}{\PYZsq{}}\PY{l+s+s1}{comp.graphics}\PY{l+s+s1}{\PYZsq{}}\PY{p}{,} \PY{l+s+s1}{\PYZsq{}}\PY{l+s+s1}{soc.religion.christian}\PY{l+s+s1}{\PYZsq{}}\PY{p}{,}\PYZbs{}
                 \PY{l+s+s1}{\PYZsq{}}\PY{l+s+s1}{sci.med}\PY{l+s+s1}{\PYZsq{}}\PY{p}{]}
        \PY{n}{data}\PY{o}{=}\PY{n}{df}\PY{p}{[}\PY{n}{df}\PY{p}{[}\PY{l+s+s2}{\PYZdq{}}\PY{l+s+s2}{label}\PY{l+s+s2}{\PYZdq{}}\PY{p}{]}\PY{o}{.}\PY{n}{isin}\PY{p}{(}\PY{n}{labels}\PY{p}{)}\PY{p}{]}
        
        \PY{c+c1}{\PYZsh{} print out the full text of the first sample}
        \PY{n+nb}{print}\PY{p}{(}\PY{n}{data}\PY{p}{[}\PY{l+s+s2}{\PYZdq{}}\PY{l+s+s2}{text}\PY{l+s+s2}{\PYZdq{}}\PY{p}{]}\PY{p}{[}\PY{l+m+mi}{0}\PY{p}{]}\PY{p}{)}
\end{Verbatim}


\begin{Verbatim}[commandchars=\\\{\}]
{\color{outcolor}Out[{\color{outcolor}2}]:}                                                 text                   label
        0  From: sd345@city.ac.uk (Michael Collier)\textbackslash{}nSubj{\ldots}           comp.graphics
        1  From: ani@ms.uky.edu (Aniruddha B. Deglurkar)\textbackslash{}{\ldots}           comp.graphics
        2  From: djohnson@cs.ucsd.edu (Darin Johnson)\textbackslash{}nSu{\ldots}  soc.religion.christian
        3  From: s0612596@let.rug.nl (M.M. Zwart)\textbackslash{}nSubjec{\ldots}  soc.religion.christian
        4  From: stanly@grok11.columbiasc.ncr.com (stanly{\ldots}  soc.religion.christian
\end{Verbatim}
            
\begin{Verbatim}[commandchars=\\\{\}]
{\color{outcolor}Out[{\color{outcolor}2}]:} array(['comp.graphics', 'soc.religion.christian', 'sci.med',
               'alt.atheism'], dtype=object)
\end{Verbatim}
            
    \begin{Verbatim}[commandchars=\\\{\}]
From: sd345@city.ac.uk (Michael Collier)
Subject: Converting images to HP LaserJet III?
Nntp-Posting-Host: hampton
Organization: The City University
Lines: 14

Does anyone know of a good way (standard PC application/PD utility) to
convert tif/img/tga files into LaserJet III format.  We would also like to
do the same, converting to HPGL (HP plotter) files.

Please email any response.

Is this the correct group?

Thanks in advance.  Michael.
-- 
Michael Collier (Programmer)                 The Computer Unit,
Email: M.P.Collier@uk.ac.city                The City University,
Tel: 071 477-8000 x3769                      London,
Fax: 071 477-8565                            EC1V 0HB.


    \end{Verbatim}

    \begin{Verbatim}[commandchars=\\\{\}]
{\color{incolor}In [{\color{incolor}3}]:} \PY{c+c1}{\PYZsh{} initialize the TfidfVectorizer }
        \PY{c+c1}{\PYZsh{} set min document frequency to 5}
        
        \PY{k+kn}{from} \PY{n+nn}{sklearn}\PY{n+nn}{.}\PY{n+nn}{feature\PYZus{}extraction}\PY{n+nn}{.}\PY{n+nn}{text} \PY{k}{import} \PY{n}{TfidfVectorizer}
        \PY{k+kn}{from} \PY{n+nn}{sklearn} \PY{k}{import} \PY{n}{metrics}
        \PY{k+kn}{from} \PY{n+nn}{nltk}\PY{n+nn}{.}\PY{n+nn}{corpus} \PY{k}{import} \PY{n}{stopwords}
        
        
        \PY{c+c1}{\PYZsh{} set the min document frequency to 5}
        \PY{c+c1}{\PYZsh{} generate tfidf matrix}
        \PY{n}{tfidf\PYZus{}vect} \PY{o}{=} \PY{n}{TfidfVectorizer}\PY{p}{(}\PY{n}{stop\PYZus{}words}\PY{o}{=}\PY{l+s+s2}{\PYZdq{}}\PY{l+s+s2}{english}\PY{l+s+s2}{\PYZdq{}}\PY{p}{,}\PYZbs{}
                                     \PY{n}{min\PYZus{}df}\PY{o}{=}\PY{l+m+mi}{5}\PY{p}{)} 
        
        \PY{n}{dtm}\PY{o}{=} \PY{n}{tfidf\PYZus{}vect}\PY{o}{.}\PY{n}{fit\PYZus{}transform}\PY{p}{(}\PY{n}{data}\PY{p}{[}\PY{l+s+s2}{\PYZdq{}}\PY{l+s+s2}{text}\PY{l+s+s2}{\PYZdq{}}\PY{p}{]}\PY{p}{)}
        \PY{n+nb}{print} \PY{p}{(}\PY{n}{dtm}\PY{o}{.}\PY{n}{shape}\PY{p}{)}
\end{Verbatim}


    \begin{Verbatim}[commandchars=\\\{\}]
(1777, 7006)

    \end{Verbatim}

    \begin{Verbatim}[commandchars=\\\{\}]
{\color{incolor}In [{\color{incolor}4}]:} \PY{c+c1}{\PYZsh{} Exercise 5.1.2 Clustering using NLTK KMean}
        
        \PY{k+kn}{from} \PY{n+nn}{nltk}\PY{n+nn}{.}\PY{n+nn}{cluster} \PY{k}{import} \PY{n}{KMeansClusterer}\PY{p}{,} \PYZbs{}
        \PY{n}{cosine\PYZus{}distance}
        
        \PY{c+c1}{\PYZsh{} set number of clusters}
        \PY{n}{num\PYZus{}clusters}\PY{o}{=}\PY{l+m+mi}{3}
        
        \PY{c+c1}{\PYZsh{} initialize clustering model}
        \PY{c+c1}{\PYZsh{} using cosine distance}
        \PY{c+c1}{\PYZsh{} clustering will repeat 20 times}
        \PY{c+c1}{\PYZsh{} each with different initial centroids}
        \PY{n}{clusterer} \PY{o}{=} \PY{n}{KMeansClusterer}\PY{p}{(}\PY{n}{num\PYZus{}clusters}\PY{p}{,} \PYZbs{}
                                    \PY{n}{cosine\PYZus{}distance}\PY{p}{,} \PYZbs{}
                                    \PY{n}{repeats}\PY{o}{=}\PY{l+m+mi}{20}\PY{p}{)}
        
        \PY{c+c1}{\PYZsh{} samples are assigned to cluster labels starting from 0}
        \PY{n}{clusters} \PY{o}{=} \PY{n}{clusterer}\PY{o}{.}\PY{n}{cluster}\PY{p}{(}\PY{n}{dtm}\PY{o}{.}\PY{n}{toarray}\PY{p}{(}\PY{p}{)}\PY{p}{,} \PYZbs{}
                                     \PY{n}{assign\PYZus{}clusters}\PY{o}{=}\PY{k+kc}{True}\PY{p}{)}
        
        \PY{c+c1}{\PYZsh{}print the cluster labels of the first 5 samples}
        \PY{n+nb}{print}\PY{p}{(}\PY{n}{clusters}\PY{p}{[}\PY{l+m+mi}{0}\PY{p}{:}\PY{l+m+mi}{5}\PY{p}{]}\PY{p}{)}
\end{Verbatim}


    \begin{Verbatim}[commandchars=\\\{\}]
[0, 0, 2, 2, 2]

    \end{Verbatim}

    \begin{Verbatim}[commandchars=\\\{\}]
{\color{incolor}In [{\color{incolor}5}]:} \PY{c+c1}{\PYZsh{} Exercise 5.1.3 Interpret each cluster by centroid}
        
        \PY{c+c1}{\PYZsh{} a centroid is the arithemtic mean }
        \PY{c+c1}{\PYZsh{} of all samples in the cluster}
        \PY{c+c1}{\PYZsh{} it may not stand for a real document}
        
        \PY{c+c1}{\PYZsh{} find top words at centroid of each cluster}
        \PY{k+kn}{from} \PY{n+nn}{sklearn} \PY{k}{import} \PY{n}{metrics}
        \PY{k+kn}{import} \PY{n+nn}{numpy} \PY{k}{as} \PY{n+nn}{np}
        
        \PY{c+c1}{\PYZsh{} clusterer.means() contains the centroids}
        \PY{c+c1}{\PYZsh{} each row is a cluster, and }
        \PY{c+c1}{\PYZsh{} each column is a feature (word)}
        \PY{n}{centroids}\PY{o}{=}\PY{n}{np}\PY{o}{.}\PY{n}{array}\PY{p}{(}\PY{n}{clusterer}\PY{o}{.}\PY{n}{means}\PY{p}{(}\PY{p}{)}\PY{p}{)}
        
        \PY{c+c1}{\PYZsh{} argsort sort the matrix in ascending order }
        \PY{c+c1}{\PYZsh{} and return locations of features before sorting}
        \PY{c+c1}{\PYZsh{} [:,::\PYZhy{}1] reverse the order}
        \PY{n}{sorted\PYZus{}centroids} \PY{o}{=} \PY{n}{centroids}\PY{o}{.}\PY{n}{argsort}\PY{p}{(}\PY{p}{)}\PY{p}{[}\PY{p}{:}\PY{p}{,} \PY{p}{:}\PY{p}{:}\PY{o}{\PYZhy{}}\PY{l+m+mi}{1}\PY{p}{]} 
        
        \PY{c+c1}{\PYZsh{} The mapping between feature (word)}
        \PY{c+c1}{\PYZsh{} index and feature (word) can be obtained by}
        \PY{c+c1}{\PYZsh{} the vectorizer\PYZsq{}s function get\PYZus{}feature\PYZus{}names()}
        \PY{n}{voc\PYZus{}lookup}\PY{o}{=} \PY{n}{tfidf\PYZus{}vect}\PY{o}{.}\PY{n}{get\PYZus{}feature\PYZus{}names}\PY{p}{(}\PY{p}{)}
        
        \PY{k}{for} \PY{n}{i} \PY{o+ow}{in} \PY{n+nb}{range}\PY{p}{(}\PY{n}{num\PYZus{}clusters}\PY{p}{)}\PY{p}{:}
            
            \PY{c+c1}{\PYZsh{} get words with top 20 tf\PYZhy{}idf weight in the centroid}
            \PY{n}{top\PYZus{}words}\PY{o}{=}\PY{p}{[}\PY{n}{voc\PYZus{}lookup}\PY{p}{[}\PY{n}{word\PYZus{}index}\PY{p}{]} \PYZbs{}
                       \PY{k}{for} \PY{n}{word\PYZus{}index} \PY{o+ow}{in} \PY{n}{sorted\PYZus{}centroids}\PY{p}{[}\PY{n}{i}\PY{p}{,} \PY{p}{:}\PY{l+m+mi}{20}\PY{p}{]}\PY{p}{]}
            \PY{n+nb}{print}\PY{p}{(}\PY{l+s+s2}{\PYZdq{}}\PY{l+s+s2}{Cluster }\PY{l+s+si}{\PYZpc{}d}\PY{l+s+s2}{:}\PY{l+s+se}{\PYZbs{}n}\PY{l+s+s2}{ }\PY{l+s+si}{\PYZpc{}s}\PY{l+s+s2}{ }\PY{l+s+s2}{\PYZdq{}} \PY{o}{\PYZpc{}} \PY{p}{(}\PY{n}{i}\PY{p}{,} \PY{l+s+s2}{\PYZdq{}}\PY{l+s+s2}{; }\PY{l+s+s2}{\PYZdq{}}\PY{o}{.}\PY{n}{join}\PY{p}{(}\PY{n}{top\PYZus{}words}\PY{p}{)}\PY{p}{)}\PY{p}{)}
\end{Verbatim}


    \begin{Verbatim}[commandchars=\\\{\}]
Cluster 0:
 edu; graphics; posting; lines; subject; host; nntp; university; organization; com; thanks; files; image; uk; file; 3d; program; help; ac; software 
Cluster 1:
 edu; com; pitt; gordon; banks; geb; msg; cs; article; writes; science; organization; subject; food; dyer; lines; doctor; don; disease; pittsburgh 
Cluster 2:
 god; edu; jesus; people; church; christians; bible; christian; christ; believe; hell; faith; think; rutgers; know; say; truth; christianity; don; subject 

    \end{Verbatim}

    \hypertarget{how-to-evaluate-clustering}{%
\subsubsection{5.2. How to evaluate
clustering}\label{how-to-evaluate-clustering}}

\begin{itemize}
\tightlist
\item
  External evaluation:

  \begin{itemize}
  \tightlist
  \item
    Obtain ``ground truth'': if data is not labeled, manually label a
    random subset of samples as ``ground truth''
  \item
    Assign each cluster to a ``true'' class by the \textbf{majority vote
    rule}, for example:
  \item
    Calculate precision and recall
  \end{itemize}

  \begin{longtable}[]{@{}ll@{}}
  \toprule
  Cluster ID & Ground Truth Class Label\tabularnewline
  \midrule
  \endhead
  0 & comp.graphics\tabularnewline
  1 & sci.med\tabularnewline
  2 & soc.religion.christian\tabularnewline
  \bottomrule
  \end{longtable}
\item
  Internal evaluation

  \begin{itemize}
  \tightlist
  \item
    Silhouette Coefficient
  \item
    Calinski-Harabaz Index
  \item
    \ldots{}
  \end{itemize}
\end{itemize}

    \begin{Verbatim}[commandchars=\\\{\}]
{\color{incolor}In [{\color{incolor}6}]:} \PY{c+c1}{\PYZsh{} Exercise 5.2.1 External evaluation}
        \PY{c+c1}{\PYZsh{} determine cluster labels and calcuate precision and recall}
        
        \PY{c+c1}{\PYZsh{} Create \PYZdq{}cluster\PYZdq{} column in the dataframe}
        \PY{n}{data}\PY{p}{[}\PY{l+s+s2}{\PYZdq{}}\PY{l+s+s2}{cluster}\PY{l+s+s2}{\PYZdq{}}\PY{p}{]}\PY{o}{=}\PY{n}{clusters}
        \PY{n}{data}\PY{o}{.}\PY{n}{head}\PY{p}{(}\PY{p}{)}
        
        \PY{c+c1}{\PYZsh{} generate crosstab between clusters and true labels}
        \PY{n}{pd}\PY{o}{.}\PY{n}{crosstab}\PY{p}{(} \PY{n}{index}\PY{o}{=}\PY{n}{data}\PY{o}{.}\PY{n}{cluster}\PY{p}{,} \PY{n}{columns}\PY{o}{=}\PY{n}{data}\PY{o}{.}\PY{n}{label}\PY{p}{)}
\end{Verbatim}


    \begin{Verbatim}[commandchars=\\\{\}]
/Users/rliu/anaconda/envs/py36/lib/python3.6/site-packages/ipykernel\_launcher.py:5: SettingWithCopyWarning: 
A value is trying to be set on a copy of a slice from a DataFrame.
Try using .loc[row\_indexer,col\_indexer] = value instead

See the caveats in the documentation: http://pandas.pydata.org/pandas-docs/stable/indexing.html\#indexing-view-versus-copy
  """

    \end{Verbatim}

\begin{Verbatim}[commandchars=\\\{\}]
{\color{outcolor}Out[{\color{outcolor}6}]:}                                                 text                   label  \textbackslash{}
        0  From: sd345@city.ac.uk (Michael Collier)\textbackslash{}nSubj{\ldots}           comp.graphics   
        1  From: ani@ms.uky.edu (Aniruddha B. Deglurkar)\textbackslash{}{\ldots}           comp.graphics   
        2  From: djohnson@cs.ucsd.edu (Darin Johnson)\textbackslash{}nSu{\ldots}  soc.religion.christian   
        3  From: s0612596@let.rug.nl (M.M. Zwart)\textbackslash{}nSubjec{\ldots}  soc.religion.christian   
        4  From: stanly@grok11.columbiasc.ncr.com (stanly{\ldots}  soc.religion.christian   
        
           cluster  
        0        0  
        1        0  
        2        2  
        3        2  
        4        2  
\end{Verbatim}
            
\begin{Verbatim}[commandchars=\\\{\}]
{\color{outcolor}Out[{\color{outcolor}6}]:} label    comp.graphics  sci.med  soc.religion.christian
        cluster                                                
        0                  575       65                      19
        1                    9      517                      13
        2                    0       12                     567
\end{Verbatim}
            
    \begin{Verbatim}[commandchars=\\\{\}]
{\color{incolor}In [{\color{incolor}7}]:} \PY{c+c1}{\PYZsh{} Map cluster id to true labels by \PYZdq{}majority vote\PYZdq{}}
        \PY{n}{cluster\PYZus{}dict}\PY{o}{=}\PY{p}{\PYZob{}}\PY{l+m+mi}{0}\PY{p}{:}\PY{l+s+s1}{\PYZsq{}}\PY{l+s+s1}{comp.graphics}\PY{l+s+s1}{\PYZsq{}}\PY{p}{,}\PYZbs{}
                      \PY{l+m+mi}{1}\PY{p}{:}\PY{l+s+s2}{\PYZdq{}}\PY{l+s+s2}{sci.med}\PY{l+s+s2}{\PYZdq{}}\PY{p}{,}\PYZbs{}
                      \PY{l+m+mi}{2}\PY{p}{:}\PY{l+s+s1}{\PYZsq{}}\PY{l+s+s1}{soc.religion.christian}\PY{l+s+s1}{\PYZsq{}}\PY{p}{\PYZcb{}}
        
        \PY{c+c1}{\PYZsh{} Map true label to cluster id}
        \PY{n}{predicted\PYZus{}target}\PY{o}{=}\PY{p}{[}\PY{n}{cluster\PYZus{}dict}\PY{p}{[}\PY{n}{i}\PY{p}{]} \PYZbs{}
                          \PY{k}{for} \PY{n}{i} \PY{o+ow}{in} \PY{n}{clusters}\PY{p}{]}
        
        \PY{n+nb}{print}\PY{p}{(}\PY{n}{metrics}\PY{o}{.}\PY{n}{classification\PYZus{}report}\PYZbs{}
              \PY{p}{(}\PY{n}{data}\PY{p}{[}\PY{l+s+s2}{\PYZdq{}}\PY{l+s+s2}{label}\PY{l+s+s2}{\PYZdq{}}\PY{p}{]}\PY{p}{,} \PY{n}{predicted\PYZus{}target}\PY{p}{)}\PY{p}{)}
\end{Verbatim}


    \begin{Verbatim}[commandchars=\\\{\}]
                        precision    recall  f1-score   support

         comp.graphics       0.87      0.98      0.93       584
               sci.med       0.96      0.87      0.91       594
soc.religion.christian       0.98      0.95      0.96       599

             micro avg       0.93      0.93      0.93      1777
             macro avg       0.94      0.93      0.93      1777
          weighted avg       0.94      0.93      0.93      1777


    \end{Verbatim}

    \hypertarget{clustering-with-sklearn-package---euclidean-distance}{%
\subsubsection{5.3. Clustering with sklearn package - Euclidean
distance}\label{clustering-with-sklearn-package---euclidean-distance}}

\begin{itemize}
\tightlist
\item
  Compare its performance with NLTK Kmeans result
\item
  Discuss: the difference between performance
\end{itemize}

    \begin{Verbatim}[commandchars=\\\{\}]
{\color{incolor}In [{\color{incolor}8}]:} \PY{c+c1}{\PYZsh{} Exercise 5.3.1 Clustering with sklearn package \PYZhy{} Euclidean distance}
        \PY{k+kn}{from} \PY{n+nn}{sklearn}\PY{n+nn}{.}\PY{n+nn}{cluster} \PY{k}{import} \PY{n}{KMeans}
        
        \PY{c+c1}{\PYZsh{} Kmeans with 20 different centroid seeds}
        \PY{n}{km} \PY{o}{=} \PY{n}{KMeans}\PY{p}{(}\PY{n}{n\PYZus{}clusters}\PY{o}{=}\PY{n}{num\PYZus{}clusters}\PY{p}{,} \PY{n}{n\PYZus{}init}\PY{o}{=}\PY{l+m+mi}{20}\PY{p}{)}\PYZbs{}
        \PY{o}{.}\PY{n}{fit}\PY{p}{(}\PY{n}{dtm}\PY{p}{)}
        \PY{n}{clusters} \PY{o}{=} \PY{n}{km}\PY{o}{.}\PY{n}{labels\PYZus{}}\PY{o}{.}\PY{n}{tolist}\PY{p}{(}\PY{p}{)}
\end{Verbatim}


    \begin{Verbatim}[commandchars=\\\{\}]
{\color{incolor}In [{\color{incolor}9}]:} \PY{c+c1}{\PYZsh{} Exercise 5.3.2 Performance Evaluation}
        
        \PY{n}{data}\PY{p}{[}\PY{l+s+s2}{\PYZdq{}}\PY{l+s+s2}{cluster}\PY{l+s+s2}{\PYZdq{}}\PY{p}{]}\PY{o}{=}\PY{n}{clusters}
        \PY{n}{data}\PY{o}{.}\PY{n}{head}\PY{p}{(}\PY{p}{)}
        
        \PY{c+c1}{\PYZsh{} generate crosstab between clusters and true labels}
        \PY{n}{pd}\PY{o}{.}\PY{n}{crosstab}\PY{p}{(} \PY{n}{index}\PY{o}{=}\PY{n}{data}\PY{o}{.}\PY{n}{cluster}\PY{p}{,} \PY{n}{columns}\PY{o}{=}\PY{n}{data}\PY{o}{.}\PY{n}{label}\PY{p}{)}
\end{Verbatim}


    \begin{Verbatim}[commandchars=\\\{\}]
/Users/rliu/anaconda/envs/py36/lib/python3.6/site-packages/ipykernel\_launcher.py:3: SettingWithCopyWarning: 
A value is trying to be set on a copy of a slice from a DataFrame.
Try using .loc[row\_indexer,col\_indexer] = value instead

See the caveats in the documentation: http://pandas.pydata.org/pandas-docs/stable/indexing.html\#indexing-view-versus-copy
  This is separate from the ipykernel package so we can avoid doing imports until

    \end{Verbatim}

\begin{Verbatim}[commandchars=\\\{\}]
{\color{outcolor}Out[{\color{outcolor}9}]:}                                                 text                   label  \textbackslash{}
        0  From: sd345@city.ac.uk (Michael Collier)\textbackslash{}nSubj{\ldots}           comp.graphics   
        1  From: ani@ms.uky.edu (Aniruddha B. Deglurkar)\textbackslash{}{\ldots}           comp.graphics   
        2  From: djohnson@cs.ucsd.edu (Darin Johnson)\textbackslash{}nSu{\ldots}  soc.religion.christian   
        3  From: s0612596@let.rug.nl (M.M. Zwart)\textbackslash{}nSubjec{\ldots}  soc.religion.christian   
        4  From: stanly@grok11.columbiasc.ncr.com (stanly{\ldots}  soc.religion.christian   
        
           cluster  
        0        0  
        1        0  
        2        1  
        3        1  
        4        1  
\end{Verbatim}
            
\begin{Verbatim}[commandchars=\\\{\}]
{\color{outcolor}Out[{\color{outcolor}9}]:} label    comp.graphics  sci.med  soc.religion.christian
        cluster                                                
        0                  584      510                     105
        1                    0        6                     494
        2                    0       78                       0
\end{Verbatim}
            
    \begin{Verbatim}[commandchars=\\\{\}]
{\color{incolor}In [{\color{incolor}10}]:} \PY{n}{cluster\PYZus{}dict}\PY{o}{=}\PY{p}{\PYZob{}}\PY{l+m+mi}{1}\PY{p}{:}\PY{l+s+s1}{\PYZsq{}}\PY{l+s+s1}{comp.graphics}\PY{l+s+s1}{\PYZsq{}}\PY{p}{,} \PY{l+m+mi}{0}\PY{p}{:}\PY{l+s+s2}{\PYZdq{}}\PY{l+s+s2}{sci.med}\PY{l+s+s2}{\PYZdq{}}\PY{p}{,}\PYZbs{}
                       \PY{l+m+mi}{2}\PY{p}{:}\PY{l+s+s1}{\PYZsq{}}\PY{l+s+s1}{soc.religion.christian}\PY{l+s+s1}{\PYZsq{}}\PY{p}{\PYZcb{}}
         
         \PY{c+c1}{\PYZsh{} Assign true class to cluster}
         \PY{n}{predicted\PYZus{}clusters}\PY{o}{=}\PY{p}{[}\PY{n}{cluster\PYZus{}dict}\PY{p}{[}\PY{n}{i}\PY{p}{]} \PY{k}{for} \PY{n}{i} \PY{o+ow}{in} \PY{n}{clusters}\PY{p}{]}
         
         \PY{n+nb}{print}\PY{p}{(}\PY{n}{metrics}\PY{o}{.}\PY{n}{classification\PYZus{}report}\PYZbs{}
               \PY{p}{(}\PY{n}{target}\PY{p}{,} \PY{n}{predicted\PYZus{}clusters}\PY{p}{)}\PY{p}{)}
\end{Verbatim}


    \begin{Verbatim}[commandchars=\\\{\}]

        ---------------------------------------------------------------------------

        NameError                                 Traceback (most recent call last)

        <ipython-input-10-97bc5e6665a6> in <module>
          7 
          8 print(metrics.classification\_report\textbackslash{}
    ----> 9       (target, predicted\_clusters))
    

        NameError: name 'target' is not defined

    \end{Verbatim}

    \begin{Verbatim}[commandchars=\\\{\}]
{\color{incolor}In [{\color{incolor} }]:} \PY{c+c1}{\PYZsh{} Exercise 5.3.3 Add \PYZdq{}alt.atheism\PYZdq{} documents to }
        \PY{c+c1}{\PYZsh{} the samples and re\PYZhy{}do clustering}
        \PY{c+c1}{\PYZsh{} Steps:}
        \PY{c+c1}{\PYZsh{} 1. add \PYZdq{}alt.atheism\PYZdq{} to line 13 of Exercise 5.1.1}
        \PY{c+c1}{\PYZsh{} 2. change the number of clusters to 4 in line 6 of Exercise 5.1.2}
        \PY{c+c1}{\PYZsh{} 3. Re\PYZhy{}run Exercise 5.1.1 and Exercise 5.1.2}
        \PY{c+c1}{\PYZsh{} 4. In Exercise 5.2.1, change the mapping between clusters }
        \PY{c+c1}{\PYZsh{}    and true class and recalculate performance}
        
        \PY{c+c1}{\PYZsh{} 5. Accordingly, try Exercise 5.3.1 and 5.3.2 to see}
        \PY{c+c1}{\PYZsh{}    the performance under Euclidean distance   }
\end{Verbatim}


    \hypertarget{how-to-pick-k-the-number-of-clusters}{%
\subsubsection{\texorpdfstring{5.4. How to pick \emph{K}, the number of
clusters?}{5.4. How to pick K, the number of clusters?}}\label{how-to-pick-k-the-number-of-clusters}}

\begin{itemize}
\tightlist
\item
  \textbf{Try external valuation first!!!}

  \begin{itemize}
  \tightlist
  \item
    manually assess a subset of documents to create ``ground truth''
  \end{itemize}
\item
  In case it is impossible to figure out how many clusters in the data
  set manually, \textbf{theorectically}, \emph{K} may be selected as
  follows:

  \begin{itemize}
  \tightlist
  \item
    Select a metric to measure the ``goodness'' of clusters,
    e.g.~average radius, average diameter, etc.
  \item
    Varying \emph{K} from 2 to N, perform clustering for each \emph{K}
  \item
    Ideally, as \emph{K} increases to some point, the metric should grow
    slowly (\textbf{elbow method})
  \end{itemize}
\end{itemize}

 source:
http://infolab.stanford.edu/\textasciitilde{}ullman/mmds/ch7.pdf -
However, if samples do not have clear structures, this method may not
work (elbow does not exist!) 


    % Add a bibliography block to the postdoc
    
    
    
    \end{document}
